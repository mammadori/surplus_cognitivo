%%%%%%%%%%%%%%%%%%%%%%%%%%%%%%%%%%%%%%%%%%%%%%%%%%%%%%%%%%%%%%%%%%%%%%%%%%%%%
%
% Copyright 2012 - Marco Amadori <marco.amadori@gmail.com>
%
% Liceza: Creative Commons 3.0 CC-BY-SA
% 
% http://creativecommons.org/licenses/by-sa/3.0/it/deed.it
%
%%%%%%%%%%%%%%%%%%%%%%%%%%%%%%%%%%%%%%%%%%%%%%%%%%%%%%%%%%%%%%%%%%%%%%%%%%%%%

\documentclass[italian,compress,red]{beamer}
\mode<presentation>{
  \usetheme{Warsaw}
  %\usecolortheme{wolverine}
  \useoutertheme[subsection=false]{smoothbars}
  \setbeamercovered{transparent}
  %\beamerdefaultoverlayspecification{<+->}
}

\usepackage[italian]{babel}
\usepackage[utf8]{inputenc}
\usepackage[T1]{fontenc}
\usepackage{times}

\usepackage{subfigure}
\usepackage{multicol}
\usepackage{amsmath}
\usepackage{epsfig}
\usepackage{graphicx}
\usepackage[all,knot]{xy}
\xyoption{arc}
\usepackage{url}
\usepackage{multimedia}
\usepackage{hyperref}
\usepackage{setspace}
\usepackage{textcomp}

\title[Software e Surplus Cognitivo]{Surplus Cognitivo, Free Software, Open Data}
\subtitle{Salire sulle spalle di tanti nani per guardare lontano}
\author[Marco Amadori]{Marco Amadori <marco.amadori@gmail.com>}
\institute{SpazioDati s.r.l --- \url{http://spaziodati.eu} \\
           Fondazione Bruno Kessler --- \url{http://www.fbk.eu}}
\date{\scriptsize Pordenone --- \vspace{.10cm}29 Settembre 2012}

\pgfdeclareimage[height=0.5cm]{ccbysa}{figures/by-sa.pdf}
\pgfdeclareimage[height=1.6cm]{logo}{figures/TEDx_Pordenone.pdf}

\pgfdeclareimage[height=1.6cm]{moglen}{figures/eben_moglen.jpg}
\pgfdeclareimage[height=1.6cm]{clay}{figures/clay-shirky.png}
\pgfdeclareimage[height=1.6cm]{newton}{figures/newton.jpg}

\pgfdeclareimage[height=1cm]{apple}{figures/Apple_logo.png}
\pgfdeclareimage[height=1cm]{ibm}{figures/IBM-logo.jpg}
\pgfdeclareimage[height=1cm]{facebook}{figures/facebook-logo.jpg}

\pgfdeclareimage[height=0.75\textheight]{cake}{figures/data-cake-graphic.jpg}
\pgfdeclareimage[height=6cm]{openbilanci}{figures/openbilanci.png}
\pgfdeclareimage[height=6cm]{bitcoinamount}{figures/Total_bitcoins_over_time_graph.png}

\pgfdeclareimage[height=0.4cm]{flitaly}{figures/flag_of_italy.png}
\pgfdeclareimage[height=0.4cm]{bitcoin}{figures/Bitcoin.png}

\logo{\pgfuseimage{ccbysa}}

\begin{document}

\begin{frame}[plain]
  \titlepage
  \begin{center}%
    \pgfuseimage{logo}%
    \end{center}%
\end{frame}

%\section[Sommario]{}
%\frame{\tableofcontents}

%ideas worth spreading
% l'unione fa la forza
% surplus cognitivo
% Software
% valore
% apple, ibm, facebook il software permette di vendere hardware, combustibile (apple ibm o struttura,facebook
% opendata e pubblica amministrazione
% data cake
% bitcoin
% finale con newton e 

%\section{L'unione fa la forza}

%Qualche anno fa, come tesi di laurea, il docente mi chiese di realizzare un progetto software abbastanza innovativo per 
%i tempi, 

%qualche anno fa mi stavo interessando per curiosità personale a come funzionano i computer quando li accendi
%certo quando la maggior parte delle persone li compra sono già pronti per partire, ma da buon smanettone
%io compravo i pezzi uno ad uno e li mettevo assieme io, poi inserivo un CD e tramite quello si 'installava' il computer
%cercando nella rete trovai dei CD che facevano funzionare un computer senza installarlo e senza toccare i dati che 
%c'erano già sul disco


\section{Surplus Cognitivo}

\begin{frame}{Surplus Cognitivo}{Tempo libero e talenti del mondo}
 \begin{itemize}
  \item tempo libero
  \item tecnologia digitale
  \item motivazione
  
  \begin{columns}[T]
  \begin{column}{.3\textwidth}
    \begin{block}{Clay Shirky}
      \begin{center}
      \pgfuseimage{clay}
      \end{center}
     \end{block}

  \end{column} 
  \begin{column}{.7\textwidth}
    \begin{exampleblock}{Il Surplus Cognitivo}
    \begin{itemize}
     \item «Rappresenta l'abilità della popolazione mondiale di contribuire volontariamente 
e collaborare a grandi progetti a volte anche globali.»
     \item  «nel mondo ci sono un milione di milardi di ore l'anno di tempo libero»
    \end{itemize}

   
      \vfill
      \begin{tiny} TED @Cannes 2010 \end{tiny}
      
    \end{exampleblock}   
  \end{column}
  \end{columns} 

 \end{itemize}

\end{frame}

\begin{frame}{Esempi}{Sforzi creativi diffusi}
\begin{columns}[T]
  \begin{column}{.5\textwidth}
   
   \begin{exampleblock}{Software Libero o Open Source}
  \begin{itemize}
    \item Linux (Android)
    \item Apache
    \item OpenOffice \begin{tiny}(LibreOffice)\end{tiny}
  \end{itemize}
  \end{exampleblock}
  \end{column}
  \pause
  \begin{column}{.5\textwidth}
   \begin{exampleblock}{Open Data}
  \begin{itemize}
    \item Wikipedia
    \item OpenStreetMap
    \item \pgfuseimage{flitaly} \hspace{0.5ex} OpenBilanci
  \end{itemize}
  \end{exampleblock}
  \end{column}

  \end{columns}

\end{frame} 
  
\section{Software}

\begin{frame}{Il Software}{l'acciaio del {21\textdegree} secolo}
  \begin{columns}[T]
  \begin{column}{.3\textwidth}
    \begin{block}{Eben Moglen}
      \begin{center}
      \pgfuseimage{moglen}
      \end{center}
      \vfill
      \begin{tiny}Legale della \\ Free Software Foundation\end{tiny}\\
    \end{block}

  \end{column} 
  \begin{column}{.7\textwidth}
    \begin{exampleblock}{Il Software}
      \begin{scriptsize}

    «\emph{Il Software} è quello di cui è fatto il 21esimo secolo. \\
    quello che l'\emph{acciaio} era per l'economia del 20esimo secolo, \\
    quello che l'\emph{acciaio} era per il potere nel 20esimo secolo, \\
    quello che l'\emph{acciaio} era per la politica nel 20esimo secolo, \\
    \emph{il Software} lo è oggi. \\
    è la pietra angolare, il componente \\ 
    sul quale tutto il resto è costruito\ldots» \\
      \end{scriptsize}
      \vfill
      \begin{tiny} Brussel --- FOSDEM 2011 \end{tiny}
      
    \end{exampleblock}   
  \end{column}
  \end{columns}

\end{frame}

\begin{frame}{ICT companies}{non funzionerebbero senza}
\begin{itemize}
  \onslide+<1-> \item \pgfuseimage{ibm}
  \onslide+<2-> \item \pgfuseimage{apple}
  \onslide+<3-> \item \pgfuseimage{facebook}
 \end{itemize}

\end{frame}

  
\section{Data}

\begin{frame}[<+->]{Free Software}{Le cose belle \underline{dell'informatica} sono gratis}
\begin{enumerate}
\setcounter{enumi}{-1}
 \item Libertà di eseguire il programma per qualsiasi scopo.
 \item Libertà di studiare il programma e modificarlo.
 \item Libertà di ridistribuire copie del programma in modo da aiutare il prossimo.
 \item Libertà di migliorare il programma e di distribuirne pubblicamente i miglioramenti, in modo tale che tutta la comunità ne tragga beneficio
\end{enumerate}

\end{frame}

\begin{frame}{Open Data}{Il cibo del Software}

\begin{center}
 \pgfuseimage{cake}
\end{center}
\end{frame}

\begin{frame}[<+->]{Open Bilanci}{Follow the money}
\begin{itemize}
 \item Quanto spende il mio comune per la manutenzione delle strade, la polizia municipale, gli asili, l'assistenza agli anziani?
 \item Raffrontato con gli altri comuni è tanto o poco?
 \item Qual è l'andamento nel tempo della spesa per questi servizi?
 \item Chi è stato sindaco in questi anni?
 \item Come è variato il bilancio comunale col passare delle varie amministrazioni? 
 \item Chi ha accumulato più debiti?
\end{itemize}
 
\end{frame}

\begin{frame}{OpenBilanci}{\url{http://opendata.comune.fi.it/open_bilancio}}
\begin{center}
 \pgfuseimage{openbilanci}
\end{center}
\end{frame}


\section{Economia}

\begin{frame}[<+->]{\pgfuseimage{bitcoin} \hspace{0.3ex} Bitcoin}{applicazione di Free Software e Open Data}
 \begin{itemize}
  \item moneta virtuale non associata ad una moneta bancaria
  \item distribuita, decentralizzata
  \item ha un costo di conio (elettricità per il calcolo)
  \item un valore (di mercato)
  \item non controllabile da uno stato o azienda
  \item software Open Source
  \item i coin sono Open Data
 \end{itemize}

\end{frame}

\begin{frame}{\pgfuseimage{bitcoin} \hspace{0.3ex} Bitcoin}{quanto ne circola?}
\begin{center}
     \pgfuseimage{bitcoinamount}
\end{center}
\end{frame}


\section{Note finali}

\begin{frame}{Conclusioni}{servono i giganti?}
  \begin{columns}[T]
  \begin{column}{.3\textwidth}
    \begin{block}{Isaac Newton}
      \begin{center}
      \pgfuseimage{newton}
      \end{center}
      \end{block}

  \end{column} 
  \begin{column}{.7\textwidth}
    \begin{exampleblock}{Il Software}
      «Se ho visto più lontano, è perché stavo sulle spalle di giganti»
      \vfill
      \begin{tiny} in una lettera a Hooke, 5 febbraio 1676\end{tiny}
      
    \end{exampleblock}   
  \end{column}
  \end{columns}
  
  \begin{itemize}
   \onslide+<2->
   \item Legge di Linus: «Dato un numero sufficiente di occhi, tutti i bug vengono a galla» 
   \onslide+<3->
   \item tanti nani volenterosi possono guardare negli occhi un gigante
  \end{itemize}


\end{frame}


\begin{frame}[<+->]{Ringraziamenti}
 \begin{itemize}
  \item Voi
  \item Giuseppe D'Orsi --- Matteo Palmisano
  \item Maurizio Napolitano -- \url{http://de.straba.us}
  \item i nani volenterosi
 \end{itemize}
 
  \begin{block}{Queste slides}
  \url{http://goo.gl/MA2Pk} [github: mammadori]
  \end{block}

\end{frame}


\end{document}

